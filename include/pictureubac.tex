\begin{frame}[allowframebreaks, fragile]{Savebox}
Spremanje slike u "kutiju" (savebox) čiji se sadržaj treba popločati:

\begin{verbatim}
  \newsavebox{ime}
  \savebox{ime}(x, y)[poz]{sadržaj}
\end{verbatim}

gdje su x i y širina i visina predviđenog prostora, a poz zastavica za pozicioniranje sadržaja na stranici.\newpage
\end{frame}

\begin{frame}[allowframebreaks, fragile]{Ubacivanje slike s ponavljajućim elementima}

Postavljanje slike naredbom \verb|\put| ili \verb|\multiput|:

\begin{verbatim}
	\put(x, y){\usebox{ime}}
\end{verbatim}

Ime kutije može se koristiti kao naredba, pa kasnije možemo prizvati sadržaj kutije koristeći znak \verb|\|prije imena.
Tako se unaprijed definira izgled grafičkih elemenata koje želimo po potrebi ponavljati.\newpage
\begin{columns}[c]

\begin{column}{0.5\textwidth}

\begin{Verbatim}[fontsize=\tiny]
\setlength{\unitlength}{0.5mm}
\begin{picture}(120,168)
\newsavebox{\foldera}
\savebox{\foldera}
  (40,32)[bl]{%
  \multiput(0,0)(0,28){2}
    {\line(1,0){40}}
  \multiput(0,0)(40,0){2}
    {\line(0,1){28}}
  \put(1,28){\oval(2,2)[tl]}
  \put(1,29){\line(1,0){5}}
  \put(9,29){\oval(6,6)[tl]}
  \put(9,32){\line(1,0){8}}
  \put(17,29){\oval(6,6)[tr]}
}

\newsavebox{\folderb}
\savebox{\folderb}
  (40,32)[l]{%
  \put(8,0){\usebox{\foldera}}
  \put(0,14){\line(1,0){8}}
}

\put(34,63){\line(0,1){65}}
\put(14,128){\usebox{\foldera}}
\multiput(34,86)(0,-37){2}
{\usebox{\folderb}}
\end{picture}
\end{Verbatim}

\end{column}

\begin{column}{0.5\textwidth}%

\setlength{\unitlength}{0.5mm}
\begin{picture}(120,168)
\newsavebox{\foldera}
\savebox{\foldera}
  (40,32)[bl]{%
  \multiput(0,0)(0,28){2}
    {\line(1,0){40}}
  \multiput(0,0)(40,0){2}
    {\line(0,1){28}}
  \put(1,28){\oval(2,2)[tl]}
  \put(1,29){\line(1,0){5}}
  \put(9,29){\oval(6,6)[tl]}
  \put(9,32){\line(1,0){8}}
  \put(17,29){\oval(6,6)[tr]}
}

\newsavebox{\folderb}
\savebox{\folderb}
  (40,32)[l]{%
  \put(8,0){\usebox{\foldera}}
  \put(0,14){\line(1,0){8}}
}

\put(34,63){\line(0,1){65}}
\put(14,128){\usebox{\foldera}}
\multiput(34,86)(0,-37){2}
{\usebox{\folderb}}
\end{picture}

\end{column}

\end{columns}

\end{frame}
