\begin{frame}[allowframebreaks, fragile]{Kvadratne Bézierove krivulje}

Kvadratne Bézierove krivulje su krivulje određene trima točkama: početnom, krajnjom i nadzornom koja leži izvan nje, te joj određuje putanju.\newline
Za crtanje kvadratnih Bézierovih krivulja ne koristimo naredbu \verb|\put|, nego \verb|\qbezier|:

\begin{verbatim}
	\qbezier(x1, y1)(x, y)(x2, y2)
\end{verbatim}

gdje koordinate prve i treće zagrade određuju početak i kraj, a one iz druge položaj nadzorne točke.\newpage
\begin{columns}[c]

\begin{column}{0.5\textwidth}

\begin{Verbatim}[fontsize=\tiny]
\setlength{\unitlength}{1cm}
\begin{picture}(6,4)
\linethickness{0.1mm}
\multiput(0,0)(1,0){7}
{\line(0,1){4}}
\multiput(0,0)(0,1){5}
{\line(1,0){6}}
\linethickness{1mm}
\put(0,2){\line(2,-1){4}}
\thicklines
\put(4,0){\line(1,3){1}}

\qbezier(0,2)(6,5)(5,3)
\qbezier(4,2)(4,3)(3,3)
\qbezier(3,3)(2,3)(2,2)
\qbezier(2,2)(2,1)(3,1)
\qbezier(3,1)(4,1)(4,2)
\end{picture}
\end{Verbatim}

\end{column}

\begin{column}{0.5\textwidth}%

\setlength{\unitlength}{1cm}
\begin{picture}(6,4)
\linethickness{0.1mm}
\multiput(0,0)(1,0){7}
{\line(0,1){4}}
\multiput(0,0)(0,1){5}
{\line(1,0){6}}
\linethickness{1mm}
\put(0,2){\line(2,-1){4}}
\thicklines
\put(4,0){\line(1,3){1}}

\qbezier(0,2)(6,5)(5,3)
\qbezier(4,2)(4,3)(3,3)
\qbezier(3,3)(2,3)(2,2)
\qbezier(2,2)(2,1)(3,1)
\qbezier(3,1)(4,1)(4,2)
\end{picture}

\end{column}

\end{columns}

\end{frame}

\begin{frame}[allowframebreaks, fragile]{Parabole}

Preko izračunatih koordinata kontrolnih točaka crtaju se dvije Bézierove krivulje koje su međusobno simetrične.

\begin{columns}[c]

\begin{column}{0.5\textwidth}

\begin{Verbatim}[fontsize=\tiny]
\setlength{\unitlength}{1cm}
\begin{picture}(4.3,3.6)(-2.5,-0.25)
\put(-2,0){\vector(1,0){4.4}}
\put(2.45,-.05){$x$}
\put(0,0){\vector(0,1){3.2}}
\put(0,3.35){\makebox(0,0){$y$}}
\qbezier(0.0,0.0)(1.2384,0.0)
(2.0,2.7622)
\qbezier(0.0,0.0)(-1.2384,0.0)
(-2.0,2.7622)
\linethickness{.075mm}
\multiput(-2,0)(1,0){5}
{\line(0,1){3}}
\multiput(-2,0)(0,1){4}
{\line(1,0){4}}
\linethickness{.2mm}
\end{picture}
\end{Verbatim}

\end{column}

\begin{column}{0.5\textwidth}%

\setlength{\unitlength}{1cm}
\begin{picture}(4.3,3.6)(-2.5,-0.25)
\put(-2,0){\vector(1,0){4.4}}
\put(2.45,-.05){$x$}
\put(0,0){\vector(0,1){3.2}}
\put(0,3.35){\makebox(0,0){$y$}}
\qbezier(0.0,0.0)(1.2384,0.0)
(2.0,2.7622)
\qbezier(0.0,0.0)(-1.2384,0.0)
(-2.0,2.7622)
\linethickness{.075mm}
\multiput(-2,0)(1,0){5}
{\line(0,1){3}}
\multiput(-2,0)(0,1){4}
{\line(1,0){4}}
\linethickness{.2mm}

\end{picture}

\end{column}

\end{columns}

\end{frame}

\begin{frame}[allowframebreaks, fragile]{Ostali grafovi}

Na sličan način mogu se iscrtavati grafovi svakakvih funkcija:

\begin{columns}[c]

\begin{column}{0.5\textwidth}

\begin{Verbatim}[fontsize=\tiny]
\setlength{\unitlength}{1cm}
\begin{picture}(6,6)(-3,-3)
\put(-1.5,0){\vector(1,0){3}}
\put(2.7,-0.1){$\chi$}
\put(0,-1.5){\vector(0,1){3}}
\multiput(-2.5,1)(0.4,0){13}
{\line(1,0){0.2}}
\multiput(-2.5,-1)(0.4,0){13}
{\line(1,0){0.2}}
\put(0.2,1.4)
{$\beta=v/c=\tanh\chi$}
\qbezier(0,0)(0.8853,0.8853)
(2,0.9640)
\qbezier(0,0)(-0.8853,-0.8853)
(-2,-0.9640)
\end{picture}
\end{Verbatim}

\end{column}

\begin{column}{0.5\textwidth}%
\setlength{\unitlength}{1cm}
\begin{picture}(6,6)(-3,-3)
\put(-2.5,0){\vector(1,0){5}}
\put(2.7,-0.1){$\chi$}
\put(0,-1.5){\vector(0,1){3}}
\multiput(-2.5,1)(0.4,0){13}
{\line(1,0){0.2}}
\multiput(-2.5,-1)(0.4,0){13}
{\line(1,0){0.2}}
\put(0.2,1.4)
{$y=\tanh\theta$}
\qbezier(0,0)(0.8853,0.8853)
(2,0.9640)
\qbezier(0,0)(-0.8853,-0.8853)
(-2,-0.9640)

\end{picture}

\end{column}

\end{columns}

\end{frame}
