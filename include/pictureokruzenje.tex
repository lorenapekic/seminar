\section{Linije}

Linije se crtaju pomoću naredbe:
\verb|\put(x, y){ \line(x1, y1){dužina} }|


\begin{center}
\verb|\line| naredba sadrži dva argumenta:

1. vektor smjera \\
2. dužina ( ovaj argument je vertikalna duljina u slučaju vertikalnog segmenta, a u svim drugim slučajevima vodoravna udaljenost linije, a ne dužina samog segmenta )
\end{center}



\justify
Komponente pravca vektora ograničene su na cijele brojeve (\verb|−6, −5, ..., 5, 6|) i moraju biti proste (nema zajedničkog djelitelja osim 1). Duljina je relativna u odnosu na postavljenu duljinu \verb|\unitlength|. 
Donja slika prikazuje svih 25 mogućih vrijednosti nagiba u prvom kvadrantu, dok naredbe prikazuju samo neke od primjera tih linija.  \\

\setlength{\unitlength}{3cm}
\begin{picture}(1,1)
\put(0,0){\line(0,1){1}}
\put(0,0){\line(1,0){1}}
\put(0,0){\line(1,1){1}}
\put(0,0){\line(1,2){.5}}
\put(0,0){\line(1,3){.3333}}
\put(0,0){\line(1,4){.25}}
\put(0,0){\line(1,5){.2}}
\put(0,0){\line(1,6){.1667}}
\put(0,0){\line(2,1){1}}
\put(0,0){\line(2,3){.6667}}
\put(0,0){\line(2,5){.4}}
\put(0,0){\line(3,1){1}}
\put(0,0){\line(3,2){1}}
\put(0,0){\line(3,4){.75}}
\put(0,0){\line(3,5){.6}}
\put(0,0){\line(4,1){1}}
\put(0,0){\line(4,3){1}}
\put(0,0){\line(4,5){.8}}
\put(0,0){\line(5,1){1}}
\put(0,0){\line(5,2){1}}
\put(0,0){\line(5,3){1}}
\put(0,0){\line(5,4){1}}
\put(0,0){\line(5,6){.8333}}
\put(0,0){\line(6,1){1}}
\put(0,0){\line(6,5){1}}
\end{picture}
\begin{flushright}


\begin{verbatim}
\setlength{\unitlength}{5cm}
\begin{picture}(1,1)
\put(0,0){\line(0,1){1}}
\put(0,0){\line(1,6){.1667}}
\put(0,0){\line(2,3){.6667}}
\put(0,0){\line(2,5){.4}}
\put(0,0){\line(3,1){1}}
\put(0,0){\line(3,4){.75}}
\end{picture}
\end{verbatim}

\end{flushright}
\newpage
\section{Strelice}

Strelice se crtaju naredbom:
\verb|\put(x, y){\vector(x1, y1){length}}|


\par
\justify
Za strelice su komponente vektorskog smjera još uže ograničene nego za crte, to jest na cijele brojeve (-4, -3, ..., 3, 4), te također moraju biti prosti.

\setlength{\unitlength}{0.75mm}
\begin{picture}(60,40)
\put(30,20){\vector(1,0){30}}
\put(30,20){\vector(4,1){20}}
\put(30,20){\vector(3,1){25}}
\put(30,20){\vector(2,1){30}}
\put(30,20){\vector(1,2){10}}
\thicklines
\put(30,20){\vector(-4,1){30}}
\put(30,20){\vector(-1,4){5}}
\thinlines
\put(30,20){\vector(-1,-1){5}}
\put(30,20){\vector(-1,-4){5}}
\end{picture}

\begin{verbatim}

\setlength{\unitlength}{0.75mm}
\begin{picture}(60,40)
\put(30,20){\vector(1,0){30}}
\put(30,20){\vector(4,1){20}}
\put(30,20){\vector(3,1){25}}
\put(30,20){\vector(2,1){30}}
\put(30,20){\vector(1,2){10}}
\thicklines
\put(30,20){\vector(-4,1){30}}
\put(30,20){\vector(-1,4){5}}
\thinlines
\put(30,20){\vector(-1,-1){5}}
\put(30,20){\vector(-1,-4){5}}
\end{picture}
\end{verbatim}

\newpage
\section{Kružnice}



\justify
Naredbom
\verb|\put(x, y){\circle{promjer}}| crta se kružnica sa središtem (x, y) i promjerom opisanim u vitičastim zagradama. Okruženje slike ( ili \textit{The picture environment}) samo dopušta promjere do otprilike 14mm, iako nisu ni svi promjeri ispod te granice mogući.\\ Naredba \verb|\circle*| proizvodi diskove, odnosno ispunjene kružnice. Kao i kod linija možda će biti potrebno kod nekih kružnica koristiti dodatne pakete kao npr. \textit{eepic}, \textit{pstricks}, ili \textit{tikz}.

\setlength{\unitlength}{2mm}
\begin{picture}(60, 40)
\put(20,30){\circle{3}}
\put(20,30){\circle{16}}
\put(20,30){\circle{32}}
\put(40,30){\circle*{1}}
\put(40,30){\circle{4}}
\put(40,30){\circle{9}}
\put(25,20){\circle*{1}}
\put(30,20){\circle*{2}}
\put(35,20){\circle*{8}}
\end{picture}



\begin{verbatim}
\setlength{\unitlength}{2mm}
\begin{picture}(60, 40)
\put(20,30){\circle{3}}
\put(20,30){\circle{16}}
\put(20,30){\circle{32}}
\put(40,30){\circle*{1}}
\put(40,30){\circle{4}}
\put(40,30){\circle{9}}
\put(25,20){\circle*{1}}
\put(30,20){\circle*{2}}
\put(35,20){\circle*{8}}
\end{picture}
\end{verbatim}

\newpage
\section{Tekst i formule}

Formule i tekst mogu se dodati u okruženje pomoću naredbe \verb|\put| gdje se također trebaju odrediti koordinate \verb|( x, y )| te se unutar vitičastih zagrada piše tekst koji želimo vidjeti na tom određenom mjestu.

\setlength{\unitlength}{0.8cm}
\begin{picture}(6,5)
\thicklines
\put(1,0.5){\line(2,1){3}}
\put(4,2){\line(-2,1){2}}
\put(2,3){\line(-2,-5){1}}
\put(0.7,0.3){$A$}
\put(4.05,1.9){$B$}
\put(1.7,2.95){$C$}
\put(3.1,2.5){$a$}
\put(1.3,1.7){$b$}
\put(2.5,1.05){$c$}
\put(0.3,4){$P=\frac{a \cdot v_a}{2}$}
\put(5,0.4){$P = \frac{a^2 \sin \beta \sin \gamma}{2 \sin \alpha} $}
\end{picture}

\begin{verbatim}
\setlength{\unitlength}{0.8cm}
\begin{picture}(6,5)
\thicklines
\put(1,0.5){\line(2,1){3}}
\put(4,2){\line(-2,1){2}}
\put(2,3){\line(-2,-5){1}}
\put(0.7,0.3){$A$}
\put(4.05,1.9){$B$}
\put(1.7,2.95){$C$}
\put(3.1,2.5){$a$}
\put(1.3,1.7){$b$}
\put(2.5,1.05){$c$}
\put(0.3,4){$P=\frac{a \cdot v_a}{2}$}
\put(5,0.4){$P = \frac{a^2 \sin \beta \sin \gamma}{2 \sin \alpha} $}
\end{picture}
\end{verbatim}
