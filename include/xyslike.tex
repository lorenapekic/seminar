\begin{frame}[allowframebreaks, fragile]{Uvod u Xy-pic}

Xy-pic je paket za slaganje grafikona i crtanje dijagrama koji se koristi principom logičke kompozicije vizualnih komponenti.\newline

Strukturiran je kao nekoliko modula, od kojih svaki definira notaciju za određenu vrstu grafičkog objekta ili strukture.\newline

Primjeri objekata su strelice, krivulje i okviri. Ti objekti mogu biti organizirani u matrici, usmjerenom grafikonu, stazi, poligonu, čvorovima i strukturi 2-ćelije.\newpage
Da bi se Xy-pic koristio, treba se dodati sljedeći paket u preambulu dokumenta:

\begin{verbatim}
\usepackage[all]{xy}
\end{verbatim}
\textbf{All} znači da se učitava veliki standardni skup funkcija iz \textit{Xy-pic}, pogodne za razvoj ovakve vrste dijagrama o kojima se ovdje raspravlja. \newline

Glavni način crtanja \textit{Xy-pic dijagrama} je preko matričnog platna, gdje je svaki element dijagrama smješten u slot matrice kao što se vidi u sljedećim primjerima.
\end{frame}

\begin{frame}[allowframebreaks, fragile]{Xy-pic}

\begin{columns}[c]

\begin{column}{0.5\textwidth}

\begin{verbatim}
\begin{displaymath}
    \xymatrix{ A  & B \\
               D  & C }
\end{displaymath}
\end{verbatim}

\end{column}

\begin{column}{0.5\textwidth}%

\begin{displaymath}
    \xymatrix{ A  & B \\
               D  & C  }
\end{displaymath}

\end{column}

\end{columns}
\mbox{}
\newline

\verb|\xymatric| naredba mora biti korištena u matematičkom okruženju. Ovdje smo odredili dva retka i dva stupca.\newpage
\end{frame}

\begin{frame}[allowframebreaks, fragile]{Strelice u Xy-matrici}

 Da bi matrica bila dijagram moramo dodati strelice pomoću \verb|\ar[]| naredbe.\newline

 Argumenti unutar uglatih zagrada su smjer kojim strelica treba pokazivati (gore - u, dolje - d, desno - r i lijevo - l).

\begin{columns}[c]

\begin{column}{0.5\textwidth}

\begin{Verbatim}[fontsize=\tiny]
\begin{displaymath}
    \xymatrix{
        A \ar[d]  \ar[r] & B  \\
        D                       & C \ar[u] \ar[l] }
\end{displaymath}
\end{Verbatim}

\end{column}

\begin{column}{0.5\textwidth}%

\begin{displaymath}
    \xymatrix{
        A \ar[d] \ar[r] & B \\
        D    & C \ar[u] \ar[l] }
\end{displaymath}

\end{column}

\end{columns}

\end{frame}

\begin{frame}[allowframebreaks, fragile]{Dijagonale i oznake}

Da bi napravili dijagonale, potrebno je koristiti više od jednog smjera.\newline

To se postiže kombiniranjem argumenata, npr. dolje desno - 'dr', gore lijevo - 'ul' itd.\newline

Ponavljanjem naredbe za smjer može se napraviti veća strelica.

\begin{columns}[c]

\begin{column}{0.5\textwidth}

\begin{Verbatim}[fontsize=\tiny]
\begin{displaymath}
    \xymatrix{
        A & B & C & D \\
        E \ar[u] \ar[ur] \ar[urr] \ar[urrr]      &  &  & }
\end{displaymath}
\end{Verbatim}

\end{column}

\begin{column}{0.5\textwidth}%

\begin{displaymath}
    \xymatrix{
        A & B & C & D \\
        E \ar[u] \ar[ur] \ar[urr] \ar[urrr]      &  &  & }
\end{displaymath}

\end{column}

\end{columns}

\newpage

Možemo nacrtati i još zanimljivije dijagrame dodavanjem oznaka strelicama.\newline

Da bismo to učinili, koristimo uobičajene naredbe za eksponent i indeks.\newline

Također je moguće staviti oznaku unutar strelice pomoću okomite crte.\newpage

\begin{columns}[c]

\begin{column}{0.5\textwidth}

\begin{Verbatim}[fontsize=\tiny]

\begin{verbatim}
\begin{displaymath}
    \xymatrix{
        A \ar[d]_g \ar[r]^{f} & D \ar[d]^{g'} & \ar[l]|d E \\
        B \ar[r]_{f'}       & C \ar[r]|b &  F \ar[u]|a  }
\end{displaymath}
\end{Verbatim}

\end{column}

\begin{column}{0.5\textwidth}%

\begin{displaymath}
    \xymatrix{
        A \ar[d]_g \ar[r]^{f} & D \ar[d]^{g'} & \ar[l]|d E \\
        B \ar[r]_{f'}       & C \ar[r]|b &  F \ar[u]|a }
\end{displaymath}

\end{column}

\end{columns}

\end{frame}

\begin{frame}[allowframebreaks, fragile]{Izgled strelica}

Strelicama u xy-matrici možemo mijenjati izgled kako bi ih lakše raspoznavali.\newline

U naredbi \verb|@stil{rep tijelo glava}| \textit{stil} može predstavljati \verb|_, ^, 2 ili 3| koji mijenjaju izgled tijela strelice, dok\newline

\textit{rep, tijelo i glava} predstavljaju izgled pojedinog dijela strelice kao u sljedećim primjerima.

\begin{columns}[c]

\begin{column}{0.5\textwidth}

\begin{Verbatim}[fontsize=\tiny]
\begin{displaymath}
    \xymatrix{
        \bullet\ar@_{<->}[rr]     && \bullet\\
        \bullet\ar@^{>.<}[rr]     && \bullet\\
        \bullet\ar@{~)}[rr]     && \bullet\\
        \bullet\ar@{=(}[rr]     && \bullet\\
        \bullet\ar@{||~/}[rr]     && \bullet\\
        \bullet\ar@{^{(}->}[rr] && \bullet\\
        \bullet\ar@2{->}[rr]    && \bullet\\
        \bullet\ar@3{~>}[rr]    && \bullet }
\end{displaymath}
\end{Verbatim}

\end{column}

\begin{column}{0.5\textwidth}%

\begin{displaymath}
    \xymatrix{
        \bullet\ar@_{<->}[rr]     && \bullet\\
        \bullet\ar@^{>.<}[rr]     && \bullet\\
        \bullet\ar@{~)}[rr]     && \bullet\\
        \bullet\ar@{=(}[rr]     && \bullet\\
        \bullet\ar@{||~/}[rr]     && \bullet\\
        \bullet\ar@{^{(}->}[rr] && \bullet\\
        \bullet\ar@2{->}[rr]    && \bullet\\
        \bullet\ar@3{~>}[rr]    && \bullet }
\end{displaymath}

\end{column}

\end{columns}

\end{frame}

\begin{frame}[allowframebreaks, fragile]{Zakrivljenost strelica}

Xy-pic nudi mnogo načina na koje se može utjecati na zakrivljenje strelica.\newline

Može se dodati dimenzija "pc" koja se umeće odmah nakon \verb| ^ | ili \verb| _ | ako je poželjna veća ili manja zakrivljenost.\newpage

Također se može u naredbi \verb|\ar@(<izlaz>,<ulaz>)| navesti argument (u, d, l ili r) u \verb|<izlaz>| kojim određujemo na kojoj će strani iz elementa "izlaziti" odnosno početi strelica, te u \verb|<ulaz>| na kojoj će strani "ulaziti" strelica u element. 

\begin{columns}[c]

\begin{column}{0.5\textwidth}

\begin{Verbatim}[fontsize=\tiny]
$$
    \xymatrix{
        \bullet \ar@/^1pc/[r]
        \ar@/_2pc/@{.>}[r] &
        \bullet }
$$

\end{Verbatim}

\end{column}

\begin{column}{0.5\textwidth}%

$    
    \xymatrix{
        \bullet \ar@/^1pc/[r]
        \ar@/_2pc/@{.>}[r] &
        \bullet }      
$
 
\end{column}

\end{columns}

\begin{columns}[c]

\begin{column}{0.5\textwidth}

\begin{Verbatim}[fontsize=\tiny]
$
    \xymatrix{
        \bullet \ar@(r,u)[r]
        \ar@/_2pc/@(dr,dl)[r] &
        \bullet }
$

\end{Verbatim}

\end{column}

\begin{column}{0.5\textwidth}%

$
    \xymatrix{
        \bullet \ar@(r,u)[r]
        \ar@/_2pc/@(dr,dl)[r] &
        \bullet }
$
 
\end{column}

\end{columns}

\end{frame}
