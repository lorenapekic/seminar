\section{Ovali}

Naredba
\verb|\put(x, y){\oval(w, h)}|
ili
\verb|\put(x, y){\oval(w, h)[pozicija]}|
stvaraju oval koji je centriran na (x,y) točki i ima širinu \textbf{w} i visinu \textbf{h}. Argumenti b, t, l, r označuju dolje, gore, lijevo, desno i mogu se kombinirati. Jedinice za ovaj parametar postavljaju se po duljini \verb|\setlength{\unitlength}{1cm}|. Drugi parametar nije obavezan i uspostavlja koordinate donjeg lijevog kuta.

Debljina crta se može kontrolirati uz pomoć naredbi \verb|\linethickness{''lenght''} ili sa \thinlines i \thicklines.|
Dok \verb|\linethickness{''lenght''}| se odnosi samo na ravne crte to jest horizontalne i vertikalne \verb|\thinlines i \thicklines| se odnosi na sve vrste crta pa čak i na ovale i krugove.

\setlength{\unitlength}{0.75cm}
\begin{picture}(6,4)
\linethickness{0.075mm}
\multiput(0,0)(1,0){7}%
{\line(0,1){4}}
\multiput(0,0)(0,1){5}%
{\line(1,0){6}}
\thicklines
\put(2,3){\oval(3,1.8)}
\thinlines
\put(3,2){\oval(3,1.8)}
\thicklines
\put(2,1){\oval(3,1.8)[tl]}
\put(4,1){\oval(3,1.8)[b]}
\put(4,3){\oval(3,1.8)[r]}
\put(3,1.5){\oval(1.8,0.4)}
\end{picture}

\begin{verbatim}
\setlength{\unitlength}{0.75cm}
\begin{picture}(6,4)
\linethickness{0.075mm}
\multiput(0,0)(1,0){7}%
{\line(0,1){4}}
\multiput(0,0)(0,1){5}%
{\line(1,0){6}}
\thicklines
\put(2,3){\oval(3,1.8)}
\thinlines
\put(3,2){\oval(3,1.8)}
\thicklines
\put(2,1){\oval(3,1.8)[tl]}
\put(4,1){\oval(3,1.8)[b]}
\put(4,3){\oval(3,1.8)[r]}
\put(3,1.5){\oval(1.8,0.4)}
\end{picture}
\end{verbatim}


\verb|\put (6,2.2) {\oval (4,2) [R]}|
naredba će nacrtati ovalni centar u točki 4,2. Parametar \verb|[r]| nije obavezan, može se koristiti r, l, t i b za prikaz desnog, lijevog, gornjeg ili donjeg dijela ovala. Ako ga nema, nacrtat će se cijeli oval.

Naredba \verb|\put(2,2.2){\circle{2}}|
crta kružnicu centriranu u točki (2,2,2) i čiji je promjer 2.