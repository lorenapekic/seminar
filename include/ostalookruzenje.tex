\begin{frame}[allowframebreaks, fragile]{Kombiniranje crta, točaka i teksta}

\begin{columns}[c]

\begin{column}{0.5\textwidth}

\begin{Verbatim}[fontsize=\tiny]
\setlength{\unitlength}{0.6cm}
\begin{picture}(12,4)
\thicklines
\put(8,3.3){{\footnotesize $3$-strana}}
\put(9,3){\circle*{0.1}}
\put(8.3,2.9){$a_2$}
\put(8,1){\circle*{0.1}}
\put(7.7,0.5){$a_0$}
\put(10,1){\circle*{0.1}}
\put(9.7,0.5){$a_1$}
\put(11,1.66){\circle*{0.1}}
\put(11.1,1.5){$a_3$}
\put(9,3){\line(3,-2){2}}
\put(10,1){\line(3,2){1}}
\put(8,1){\line(1,0){2}}
\put(8,1){\line(1,2){1}}
\put(10,1){\line(-1,2){1}}
\end{picture}
\end{Verbatim}

\end{column}

\begin{column}{0.5\textwidth}%

\setlength{\unitlength}{0.6cm}
\begin{picture}(12,4)(5,0)
\thicklines
\put(8,3.3){{\footnotesize $3$-strana}}
\put(9,3){\circle*{0.1}}
\put(8.3,2.9){$a_2$}
\put(8,1){\circle*{0.1}}
\put(7.7,0.5){$a_0$}
\put(10,1){\circle*{0.1}}
\put(9.7,0.5){$a_1$}
\put(11,1.66){\circle*{0.1}}
\put(11.1,1.5){$a_3$}
\put(9,3){\line(3,-2){2}}
\put(10,1){\line(3,2){1}}
\put(8,1){\line(1,0){2}}
\put(8,1){\line(1,2){1}}
\put(10,1){\line(-1,2){1}}
\end{picture}

\end{column}

\end{columns}
\mbox{}
\newpage
\footnotesize{
Više jednostavnih elemenata tvori složeniju sliku. U prethodnom primjeru nekoliko crta i točaka spojeno je međusobno da stvori sliku, zatim dodajemo tekst kojim obilježavamo svaki element slike. Sljedeće su naredbe korištene:\newline

\verb|\thicklines|\newline
Ovo mijenja debljinu crta, čineći ih malo debljima. Isto tako možemo koristiti \verb|\thinlines| naredbu sa suprotnim učinkom.\newline

\verb|\put(8,3.3){{\footnotesize $3$-strana}}|\newline
Tekst "3-strana" je umeće u točku (8,3,3), veličina slova je namještena na \textit{footnotesize}. Ista naredba koristi se da označi svaku točku.\newline

\verb|\put(9,3){\circle*{0.1}}|\newline
Ova naredba crta puni krug koji je centriran u točki (9,3) i ima promjer 0.1. Promjer je tako malen da se kružnica koristi za označavanje točke.\newline

\verb|\put(10,1){\line(3,2){1}}|\newline
Crta ravnu crtu čiji je početak u točki (10,1). Dužina joj je 1, a smjer (3,2). Kao što se može vidjeti, crte, to jest pravci s nagibom teži su za nacrtati od običnih crta i pravaca.}\newpage
Vektori ili strelice mogu se isto tako korisiti u slikama, evo primjera:

\begin{columns}[c]

\begin{column}{0.5\textwidth}

\begin{Verbatim}[fontsize=\tiny]
\setlength{\unitlength}{0.20mm}
\begin{picture}(400,250)
\put(75,10){\line(1,0){130}}
\put(75,50){\line(1,0){130}}
\put(75,200){\line(1,0){130}}
\put(120,200){\vector(0,-1){150}}
\put(190,200){\vector(0,-1){190}}
\put(97,120){$\alpha$}
\put(170,120){$\beta$}
\put(220,195){više stanje}
\put(220,45){niže stanje 1}
\put(220,5){niže stanje 2}
\end{picture}
\end{Verbatim}

\end{column}

\begin{column}{0.5\textwidth}%

\setlength{\unitlength}{0.20mm}
\begin{picture}(400,250)(5,0)
\put(75,10){\line(1,0){130}}
\put(75,50){\line(1,0){130}}
\put(75,200){\line(1,0){130}}
\put(120,200){\vector(0,-1){150}}
\put(190,200){\vector(0,-1){190}}
\put(97,120){$\alpha$}
\put(170,120){$\beta$}
\put(220,195){više stanje}
\put(220,45){niže stanje 1}
\put(220,5){niže stanje 2}
\end{picture}

\end{column}

\end{columns}

\end{frame}
