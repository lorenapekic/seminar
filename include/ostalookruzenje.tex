\section{Kombiniranje crta, točaka i teksta}

Više jednostavnih elemenata tvori složeniju sliku.

\setlength{\unitlength}{0.8cm}
\begin{picture}(12,4)
\thicklines
\put(8,3.3){{\footnotesize $3$-simplex}}
\put(9,3){\circle*{0.1}}
\put(8.3,2.9){$a_2$}
\put(8,1){\circle*{0.1}}
\put(7.7,0.5){$a_0$}
\put(10,1){\circle*{0.1}}
\put(9.7,0.5){$a_1$}
\put(11,1.66){\circle*{0.1}}
\put(11.1,1.5){$a_3$}
\put(9,3){\line(3,-2){2}}
\put(10,1){\line(3,2){1}}
\put(8,1){\line(1,0){2}}
\put(8,1){\line(1,2){1}}
\put(10,1){\line(-1,2){1}}
\end{picture}

\begin{verbatim}
\setlength{\unitlength}{0.8cm}
\begin{picture}(12,4)
\thicklines
\put(8,3.3){{\footnotesize $3$-simplex}}
\put(9,3){\circle*{0.1}}
\put(8.3,2.9){$a_2$}
\put(8,1){\circle*{0.1}}
\put(7.7,0.5){$a_0$}
\put(10,1){\circle*{0.1}}
\put(9.7,0.5){$a_1$}
\put(11,1.66){\circle*{0.1}}
\put(11.1,1.5){$a_3$}
\put(9,3){\line(3,-2){2}}
\put(10,1){\line(3,2){1}}
\put(8,1){\line(1,0){2}}
\put(8,1){\line(1,2){1}}
\put(10,1){\line(-1,2){1}}
\end{picture}
\end{verbatim}


U ovom primjeru nekoliko crta i točaka je spojeno međusobno da stvori sliku, zatim dodajemo tekst kojime obilježavamo svaki element slike. Sljedeće su naredbe korištene:

\verb|\thicklines|
Ovo mijenja debljinu crta, čineći ih malo debljima isto tako možemo koristiti \verb|\thinlines| naredbu koja ima suprotni učinak.

\verb|\put(8,3.3){{\footnotesize $3$-simplex}}|
Tekst "3-simplex" je umeće u točku (8,3,3), veličina slova je namještena na \textit{footnotesize}. Ista naredba se koristi da označi svaku točku.

\verb|\put(9,3){\circle*{0.1}}|
Ova naredba crta puni krug koji je centriran u točki (9,3) i ima promjer 0.1. Promjer je tako malen da se kružnica koristi za označavanje točke.

\verb|\put(10,1){\line(3,2){1}}|
Crta ravnu crtu čiji početak je u točki (10,1), dužina je 1, a smjer je (3,2). Kao što se može vidjeti crte to jest pravci koji imaju nagib su teži za nacrtati od običnih crta i pravaca.

Vektori ili strelice se mogu korisiti isto tako u slikama, evo primjera:

\setlength{\unitlength}{0.20mm}
\begin{picture}(400,250)
\put(75,10){\line(1,0){130}}
\put(75,50){\line(1,0){130}}
\put(75,200){\line(1,0){130}}
\put(120,200){\vector(0,-1){150}}
\put(190,200){\vector(0,-1){190}}
\put(97,120){$\alpha$}
\put(170,120){$\beta$}
\put(220,195){upper state}
\put(220,45){lower state 1}
\put(220,5){lower state 2}
\end{picture}

\begin{verbatim}
\setlength{\unitlength}{0.20mm}
\begin{picture}(400,250)
\put(75,10){\line(1,0){130}}
\put(75,50){\line(1,0){130}}
\put(75,200){\line(1,0){130}}
\put(120,200){\vector(0,-1){150}}
\put(190,200){\vector(0,-1){190}}
\put(97,120){$\alpha$}
\put(170,120){$\beta$}
\put(220,195){upper state}
\put(220,45){lower state 1}
\put(220,5){lower state 2}
\end{picture}
\end{verbatim}


\section{Bezierove krivulje}

\textit{Bezierove krivulje} su posebna vrsta krivulja koje se crtaju koristeći 3 različita parametra, jedna početna točka i jedna završna, a uz to još jedna točka koja određuje koliko je krivulja "iskrivljena".

\setlength{\unitlength}{0.8cm}
\begin{picture}(10,5)
\thicklines
\qbezier(1,1)(5,5)(9,0.5)
\put(2,1){{Bézier curve}}
\end{picture}

\begin{verbatim}
\setlength{\unitlength}{0.8cm}
\begin{picture}(10,5)
\thicklines
\qbezier(1,1)(5,5)(9,0.5)
\put(2,1){{Bézier curve}}
\end{picture}
\end{verbatim}

Naredba  \verb|\qbezier| je unutar \verb|\put| naredbe. Parametri koji moraju biti određeni su početna točka, točka kontrole i završna točka.
